\documentclass{article}
\usepackage[a4paper, top=3cm, left=3cm, right=2.5cm, bottom=2.5cm]{geometry}
\usepackage[utf8]{inputenc}
\usepackage{graphicx}
\usepackage{float}
\usepackage{fancyvrb}
\usepackage{amsmath}
\usepackage{ragged2e}
\usepackage[portuguese]{babel}

\begin{document}

\title{\Huge
       \textbf{UNIVERSIDADE DO MINHO}\\
       \vspace*{3cm}
       \huge
       \textbf{\textit{Health Care System}}\\
       \vspace*{3cm}
       \large
       Mestrado Integrado de Engenharia Informática\\
       \vspace*{2cm}
       Sistemas de Representação de Conhecimento e Raciocínio\\
       (2ºSemestre/2017-2018)
       \vspace*{\fill}}

\author{\hspace*{-5cm}Número\hspace*{1cm}Nome do(s) Autor(es)\hspace*{\fill}\\
        \hspace*{-5cm}a78468\hspace*{1cm}João Vieira\hspace*{\fill}\\
        \hspace*{-5cm}a78821\hspace*{1cm}José Martins\hspace*{\fill}\\
        \hspace*{-5cm}a77049\hspace*{1cm}Miguel Quaresma\hspace*{\fill}\\
        \hspace*{-5cm}a77689\hspace*{1cm}Simão Barbosa\hspace*{\fill}}

\date{\hspace*{\fill}Braga, Portugal\hspace*{1cm}\\
      \hspace*{\fill}\today\hspace*{1cm}}

\maketitle

\newpage

\justify

\vspace*{\fill}
\section{Resumo}
O objetivo deste projeto é criar um sistema de representação de conhecimento e raciocínio na caraterização de um universo na área da prestação de cuidados de saúde, em que as medidas de sucesso mínimas são as propostas pelo enunciado. Contudo foram desenvolvidas algumas funcionalidades extras para enriquecer a base de conhecimento. 
\vspace*{\fill}

\newpage

\vspace*{\fill}
\tableofcontents
\vspace*{\fill}

\newpage

\vspace*{\fill}
\section{Introdução}
\textit{Health Care System} é um Sistema de Representação de Conhecimento e Raciocínio usado na caraterização de um universo na área da prestação de cuidados de saúde desenvolvido usando a linguagem de programação em lógica PROLOG. O projeto desenvolvido apresenta funcionalidades básicas associadas ao sistema em questão tendo ainda sido complementado com outras (funcionalidades) que complementam o universo em causa.
\vspace*{\fill}

\newpage

\section{Descrição do Trabalho e Análise de Resultados}

\subsection{Base de Conhecimento}
A base de conhecimento do sistema desenvolvido é essencial à representação do conhecimento e raciocínio, como tal, tendo em conta o sistema em questão, foram desenvolvidas as seguintes entidades:
\begin{itemize}
	\item utente: \#IdUt, Nome, Idade, Morada $\to$ \{V,F\}
    \item prestador: \#IdPrest, Nome, Especialidade, Instituição $\to$ \{V,F\}
    \item cuidado: Data, \#IdUt, \#IdPrest, Descrição, Custo $\to$ \{V,F\}
    \item instituicao: \#IdIt, Nome, Tipo, Cidade $\to$ \{V,F\}
\end{itemize}
São estas as entidades que servirão de suporte ao sistema desenvolvido.

\subsection{Funcionalidades}
As regras de uma base de conhecimento conferem-lhe a utilidade necessária ao seu funcionamento visto que, se não for possível efetuar questões e obter respostas às mesmas, a utilidade da base de conhecimento é nula. Em PROLOG estas funcionalidades/regras são implementadas tendo por base as caraterísticas desta linguagem:
\begin{itemize} 
	\item Algoritmo de Resolução: mais propriamente o \textit{Modus Tollens} (\{A se B,$\neg$A\}$\vdash\neg$B) no qual uma questão é verdade se adicionando a negação da mesma à base de conhecimento origina uma inconsistência
    \item Clausulado de Horn: todas as regras são uma cláusula de Horn \textbf{i.e.} o qual admite apenas 1 termo positivo(conclusão), as fórmulas são bem formadas e fechadas, quantificadas universalmente (\textbf{e.g}. $\forall{A,B}$ avo(A,B)$\vdash$neto(B,A))
    \item Mecanismo de \textit{backtracking}: na presença de uma solução inválida(falsa) esta meta-heurística continua à procura de "outro caminho" de modo a que a regra seja verdadeira.
\end{itemize}
Adicionalmente assumem-se vários pressupostos de entre os quais:
\begin{itemize}
	\item Pressuposto dos Nomes Únicos: duas constantes designam duas entidades diferentes
    \item Pressuposto do Mundo Fechado: tudo o que não existe mencionado é falso
    \item Pressuposto do Domínio Fechado: não há mais objetos no universo para além dos designados por constantes
\end{itemize}
Por fim o sistema de representação de conhecimento e raciocínio desenvolvido deve respeitar as "leis" da lógica recorrendo para isso a invariantes que garantem que certas propriedades são respeitadas. Estes (invariantes) garantem a inexistência de inconsistências bem como a preservação do "significado do conhecimento" e são usados na evolução e involução da base de conhecimento.
\newline
Foram assim, implementadas as seguintes funcionalidades:
\begin{itemize}
	\item Registar utentes, prestadores e cuidados de saúde,instituicoes
    \newline
    registarUtente: Id, Nome, Idade, Morada $\to$ \{V,F\}
	\newline
    registarPrestador: Id, Nome, Especialidade, Instituição $\to$ \{V,F\}
	\newline
    registarCuidado: Data, IdUtente, IdPrestador, Descrição, Custo $\to$ \{V,F\}
	\newline
    registarInstituicao: Id, Nome, Tipo, Cidade $\to$ \{V,F\}
    \item Remover utentes, prestadores e cuidados de saúde, instituicoes
    \newline
    removerUtente: Id, Nome, Idade, Morada $\to$ \{V,F\}
    \newline
    removerPrestador: Id, Nome, Especialidade, Instituição $\to$ \{V,F\}
    \newline
    removerCuidado: Data, IdUtente, IdPrestador, Descrição, Custo $\to$ \{V,F\}
    \newline
    removerInstituicao: Id, Nome, Tipo, Cidade $\to$ \{V,F\}
    \item Identificar utentes por critérios de seleção
    \newline
    identificaUtente : nome, NomeUtente, Solução $\to$ \{V,F\}
    \newline
    identificaUtente : idade, IdadeUtente, Solução $\to$ \{V,F\}
    \newline
    identificaUtente : morada, MoradaUtente, Solução $\to$ \{V,F\}
    \item Identificar as instituições prestadoras de cuidados de saúde
    \newline
    identificaInstituicoes : Solução $\to$ \{V,F\}
    \item Identificar cuidados de saúde prestados por instituição/cidade/datas
    \newline
    identCuidPrest : instituicao, Instituicao, Resultado $\to$ \{V,F\}
    \newline
    identCuidPrest : cidade, Cidade, Resultado $\to$ \{V,F\}
    \newline
    identCuidPrest : datas, Data, Data, Resultado $\to$ \{V,F\}
    \item
    Identificar os utentes de um prestador/especialidade/instituição
    \newline
    identUtentes : prestador, IdPrestador, Resultado $\to$ \{V,F\}
    \newline
    identUtentes : especialidade, Especialidade, Resultado $\to$ \{V,F\}
    \newline
    identUtentes : instituicao, Instituicao, Resultado $\to$ \{V,F\}
    \item Identificar cuidados de saúde realizados por utente/instituição/prestador
    \newline
    identificaCuidadosRealizados: utente, IdUtente, Resultado $\to$ \{V,F\}
    \newline
    identificaCuidadosRealizados: instituicao, Instituição, Resultado $\to$ \{V,F\}
    \newline
    identificaCuidadosRealizados: prestador, IdPrestador, Resultado $\to$ \{V,F\}
    \item Determinar todas as instituições/prestadores a que um utente já recorreu
    \newline
    porUtente: instituicao, IdUtente, Resultado $\to$ \{V,F\}
    \newline
    porUtente: prestador, IdUtente, Resultado $\to$ \{V,F\}
    \item Calcular o custo total dos cuidados de saúde por utente/especialidade/prestador/datas
    \newline
    custoTotal: utente, IdUtente, Resultado $\to$ \{V,F\}
    \newline
    custoTotal: especialidade, Especialidade, Resultado $\to$ \{V,F\}
    \newline
    custoTotal: prestador, IdPrestador, Resultado $\to$ \{V,F\}
    \newline
    custoTotal: datas, Data, Data, Resultado $\to$ \{V,F\}
\end{itemize}

\subsubsection{Funcionalidades Extra}
Por forma a apresentar uma descrição mais fiel do sistema em causa foram desenvolvidas/implementadas funcionalidades extra que permitem uma interação mais complexa com a base de conhecimento.
\newline
Tendo em conta as entidades sugeridas no enunciado (utente, prestador e cuidado) foi considerado pelo grupo como pertinente adicionar funcionalidades que permitam obter as seguintes informações:
\begin{itemize}
	\item Determinar as especialidades com que um utente esteve relacionado, devolvendo a data das mesmas
\newline
especialidadesDeUtente : IdUtente, Solução $\to$ \{V,F\}
	\item Determinar os prestadores que cuidaram de um utente numa dada instituição
\newline
prestadoresDeUtenteEmInstituicao : IdUtente, Solução $\to$ \{V,F\}
	\item Que utente tem mais cuidados realizados
\newline
utenteComMaisCuidados: Resultado $\to$ \{V,F\}
    \item Que prestador prestou mais cuidados
\newline
prestadorComMaisCuidados: Resultado $\to$ \{V,F\}
\end{itemize}
Para além disto, a nova entidade sugerida pelo grupo (instituição) permitiu aumentar o leque de funcionalidades e de informação a ser possível recolher da base de conhecimento em causa.
\newline
Tendo isto em conta, foram implementadas as seguintes funcionalidades:
\begin{itemize}
	\item Determinar as instituições existentes numa cidade
\newline
instituicoesDeCidade: Cidade, Solução $\to$ \{V,F\}
	\item Determinar os tipos de instituições existentes numa cidade
\newline
tiposDeInstituicoesDeCidade: Cidade, Solução $\to$ \{V,F\}
	\item Determinar o tipo de instituições que um utente já visitou
\newline
tiposInstituicoesVisitadasPorUtente: IdUtente, Solução $\to$ \{V,F\}
	\item Que instituições são hospitais/clínicas/centros de saúde
\newline
instituicoesDoTipo: Tipo, Solução $\to$ \{V,F\}
	\item Em que cidade foram mais cuidados realizados
\newline
cidadeComMaisCuidados: Resultado $\to$ \{V,F\}
    \item Qual a instituição com mais cuidados realizados
\newline
instituicaoComMaisCuidados: Resultado $\to$ \{V,F\}
    \item Em que instituição o utente realizou mais cuidados
\newline
instituicaoMaisFrequentadaPorUtente: IdUtente, Resultado $\to$ \{V,F\}
\end{itemize}

\subsubsection{Invariantes}

Com a implementação de certos invariantes considerados como importantes para o contexto em causa, a base de dados de conhecimento em PROLOG fica assim mais coerente e menos suscetível a falhas.
Tendo isto em conta, foram implementados 10 invariantes que permitem que:
\begin{itemize}
	\item não existam utentes com \textit{id's} repetidos
    \item não existam prestadores com \textit{id's} repetidos
    \item não existam instituições com \textit{id's} repetidos
    \item um prestador tenha que pertencer obrigatoriamente a uma instituição da base de conhecimento
    \item não existam cuidados repetidos
    \item seja garantida a existência do utente do cuidado a adicionar
    \item seja garantida a existência do prestador do cuidado a adicionar
    \item não pode ser removido um utente com cuidados associados
    \item não pode ser removido um prestador com cuidados associados
    \item não pode ser removida uma instituição se tiver associada a prestadores
\end{itemize}


\subsubsection{Funções Auxiliares}
O desenvolvimento do sistema em causa envolveu, por vezes, o uso de regras que partilhavam certas operações entre si, como tal estas operações foram degeneradas em regras individuais por forma a reduzir a quantidade de código necessária. As operações referidas encontram-se descritas de seguida:
\begin{itemize}
	\item Encontra todos os predicados(Questão) que sejam satisfeitos ao efetuar o \textit{backtracking} tendo Formato em conta
\newline
solucoes : Formato, Questao, Soluçoes $\to$ \{V,F\}
	\item Função para somar uma lista      
\newline
somaLista: Lista,Solucao $\to$ \{V,F\}
	\item Conta o número de ocorrências de um elemento numa lista
\newline
contaOcorrencias: Elemento,Lista,Solucao $\to$ \{V,F\}
	\item Calcula o elemento mais frequente de um par
\newline
maxFreqPair: Elemento1,Frequencia,Elemento2,Frequencia,Solucao $\to$ \{V,F\}
	\item Calcula o elemento mais frequente de uma lista
\newline
maxRepeticoes: Lista,Solucao $\to$ \{V,F\}
	\item Inserir conhecimento
\newline
inserir: Termo $\to$ \{V,F\}
	\item Remover conhecimento
\newline
remover: Termo $\to$ \{V,F\}
	\item Regra de teste dos invariantes correspondentes
\newline
test: Lista $\to$ \{V,F\}
\end{itemize}

\newpage

\vspace*{\fill}
\section{Conclusões e Sugestões}
Para concluir, os objetivos propostos pelo enunciado foram cumpridos mas também foram adicionadas novas funcionalidades com o propósito de fundamentar a base de conhecimento.Ainda assim é possível adicionar novas funcionalidades extras ao trabalho apresentado para enriquecer ainda mais a \textit{Health Care System}.
\vspace*{\fill}

\newpage

\begin{thebibliography}{10}
  \bibitem{JCMLNT}
    MARTINS, José Carlos Lima,
    \textit{Notas Teóricas},
    José Carlos Lima Martins, 
    2018.
  \bibitem{JPFVNT}
    VIEIRA, João Pedro Ferreira,
    \textit{Notas Teóricas},
    João Pedro Ferreira Vieira, 
    2018.
  \bibitem{MMQNT}
    QUARESMA, Miguel Miranda,
    \textit{Notas Teóricas},
    Miguel Miranda Quaresma, 
    2018.
  \bibitem{SPLBNT}
    BARBOSA, Simão Paulo Leal,
    \textit{Notas Teóricas},
    Simão Paulo Leal Barbosa, 
    2018. 
\end{thebibliography}
\end{document}
